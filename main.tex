%Documento de Latex para Taller 2 Programación con restricciones
\documentclass{article}
\usepackage[utf8]{inputenc}
\usepackage[spanish]{babel}
\usepackage{amsfonts}
\usepackage{design_ASC}
\usepackage{caption}

\title{
    Taller 2 PPR.\\
}
\author{
    Universidad del Valle\\
    Facultad de Ingeniería\\
    Escuela de Ingeniería de Sistemas y Computación - EISC\\
    Programación por Restricciones\\
    Profesor Juan Francisco Diaz Ph.D.\\
    Estudiante Sebastián Afanador cod. 1629587
}
\date{Junio 12, 2022}

\begin{document}
\maketitle
\section{Construcción de Transformadores}
\subsection{Tabla de datos}
\begin{table}
    \begin{center}
        \begin{tabular}{ |c|c|c|c| } 
            \hline
            & Material ferromagnético (ton) & 
            Tiempo de trabajo (h) & 
            Precio (USD) \\ 
            \hline\hline
            Transformador 1 & 2 & 7 & 120000 \\ 
            Transformador 2 & 1 & 8 & 80000 \\ 
            \hline
        \end{tabular}
        \caption{Requerimientos para la fabricación de transformadores}
        \label{tab:caption}
    \end{center}
\end{table}

\subsection{Modelo 1}
\subsubsection{Parámetros}
Vease la tabla de la sección anterior.
\subsubsection{Variables}
Sea $C=\{c_1,c_2\}$ las cantidades de transformadores a fabricar por la empresa de Walter, cada transformador tiene unos requerimientos de fabricación mostrados según la tabla anterior y a su vez un precio de venta.
\subsubsection{Restricciones}
El total de material ferromagnético disponible es de 6 toneladas y el total de tiempo para fabricación es de 28 horas entonces:\\
\begin{enumerate}
    \item $7c_1 + 8c_2 <= 28$
    \item $2c_1 + c_2 <= 6$
    \item $c_1, c_2 >= 0$
\end{enumerate}
\subsubsection{Función objetivo}
$max(120000c_1 + 80000c_2)$


\subsection{Modelo 2}
Este modelo propone un número de $n$ transformadores y $m$ requerimientos para cada transformador.
\subsubsection{Parámetros}
Sea $n$ el número de transformadores disponibles para fabricación.Sea $m$ el número de requerimientos que tiene cada transformador para su fabricación. Sea $R$ los requerimientos de fabricación de cada transformador donde $r_i \in R$ con $i \in [1,m]$ es la cantidad de un requerimientos espefífico. Sea $RT_{ri}$ el requerimiento de fabricación $r$ del transformador $i$ con $i \in [1,n]$. Sean $P$ los precios de venta de cada transformador donde $p_i \in P$ con $i \in [1,n]$ es el precio de venta de un transformador específico y finalmente sea $D$ la disponibilidad de materiales para la fabricación de los transformadores donde $d_i \in D$ con $i \in [1,m]$ es la disponibilidad de un material específico.
\subsubsection{Variables}
Sea $C$ la especificación de producción de transformadores donde $c_i \in C$ con $i \in \mathbb{Z} \land i \geq  0$ es la cantidad de producción de un transformador en específico.
\subsubsection{Restricciones}
\begin{enumerate}
    \item $\forall r_{i} \in R$ con $i \in [1,m] \sum_{j=1}^n RT_{ji}*C_j \leq D_{i} $
\end{enumerate}
\subsubsection{Función objetivo}
$max(\sum_{i=1}^{n} P_i*C_i)$


\section{Producción de gasolina}
\subsection{Modelo 1}
Teniendo en cuenta que los requerimientos de algunos 
\subsubsection{Parámetros}
Una compañía petrolera desea producir $p=120000$ galones de gasolina común, esta gasolina tiene unos requerimientos de Octano, Vapor de presión y Volatilidad. La gasolina es preparada mezclando tres ingredientes diferentes que son Butano, Catalítico reformado y Nafta pesado, cada ingrediente tienen un aporte para cada uno de los requerimientos de la gasolina que al mezclarse se suman de manera directa. La siguiente tabla muestra cuál es el aporte de cada ingrediente a cada requerimiento por cada galón y los valores mínimos y máximos de cada requerimiento para producir un galón de gasolina.\\

\begin{table}
    \begin{center}
        \begin{tabular}{|c|c|c|c||c|}
            \hline
            & Butano & Catalítico reformado & Nafta pesado & Gasolina\\
            \hline\hline
            Octano              & $120$   & $100$   & $74$    & $\geq 89$   \\
            Vapor de presión    & $60$    & $2.5$   & $4.0$   & $\leq 11$   \\
            Volatilidad         & $105$   & $3$     & $12$    & $\geq 17$   \\
            \hline
        \end{tabular}
        \caption{Aporte de requerimientos por ingrediente y límites para gasolina}
        \label{tab:caption}
    \end{center}
\end{table}
\begin{table}
    \begin{center}
        \begin{tabular}{|c|c|c|c|}
            \hline
            & Butano & Catalítico reformado & Nafta pesado\\
            \hline\hline
            Precio*gl   & $\$0.58$  & $\$1.55$  & $\$0.85$\\
            \hline
        \end{tabular}
        \caption{Precios de ingrediente por galón}
        \label{tab:caption}
    \end{center}
\end{table}
\subsubsection{Variables}
Sea $C=\{c_1,c_2,c_3\}$ las cantidades de cada uno de los tres ingredientes que desean conocer el problema para mezclar y formar gasolina común a un costo mínimo.
\subsubsection{Restricciones}
\begin{enumerate}
    \item $c_1,c_2,c_3 \geq 0$
    \item $c_1+c_2+c_3=p$
    \item $120c_1+100c_2+74c_3\geq 89p$
    \item $60c_1+2.5c_2+4.0c_3\leq 11p$
    \item $105c_1+3c_2+12c_3\geq 17p$
\end{enumerate}
\subsubsection{Función objetivo}
$min(0.58c_1+1.55c_2+0.85c_3)$

\subsection{Modelo 2}
Este modelo genérico propone el uso de ingredientes dinámicos para la producción de cualquier compuesto producido a partir de la mezcla de varios ingredientes teniendo en cuenta sus aportes con respecto a una distribución de restricciones.\\

\subsubsection{Parámetros}
Sea $p$ el total de galones que se desea producir de un compuesto. Sean $A$ los ingredientes del compuesto que se desea preparar donde $a_i \in A$ con $i \in [1..n]$ es un ingrediente. Sea $AB_{ab}$ el aporte $b$ con $b \in [1,m]$ de un ingrediente $a$ con $a \in [1,n]$ para el compuesto que se desea preparar. Sean $C$ los precios donde $c_i \in C$ con $i \in [1,n]$ es el costo por galón del ingrediente $i$.\\

Sean $D$ los límites mínimos y máximos de aporte total donde $d_i \in D$ con $i \in [1,m]$ son los límites inferiones y superiores $\{g,l\}$ del aporte $i$.\\

\subsubsection{Variables}
Sean $E$ las cantidades necesarias de los ingredientes donde $e_i \in E$ con $i \in [1,n]$ es la cantidad de un ingrediente para producir el compuesto teniendo en cuenta sus aportes.\\

\subsubsection{Restricciones}
\begin{enumerate}
    \item $\sum_{i=1}^n c_i= p$
\end{enumerate}

\subsubsection{Función objetivo}
\begin{enumerate}
    \item $min(\sum_{i=1}^n c_i*e_i)$
\end{enumerate}








\end{document}